\documentclass[12pt, a4paper]{article}
\usepackage[russian]{babel}
\usepackage{fontspec}
\setsansfont{Calibri}
\setmonofont{Consolas}
\setmainfont[
    Ligatures=TeX,
    Extension=.otf,
    BoldFont=cmunbx,
    ItalicFont=cmunti,
    BoldItalicFont=cmunbi,
]{cmunrm}
\usepackage{polyglossia}
\setdefaultlanguage{russian}
\setotherlanguage{english}


\usepackage{geometry}
\usepackage{pgfplotstable}

\geometry{
margin=2cm
}

% Создаем команду, чтобы переносить текст на новую строку внутри таблицы
\newcommand{\tcell}[2][l]{\begin{tabular}[#1]{@{}c@{}}#2\end{tabular}}

\usepackage{indentfirst}

\usepackage{arydshln}
\usepackage[fleqn]{amsmath}
\usepackage{xfrac}
\usepackage{esint}
\usepackage{amssymb}
\usepackage{mathbbol}
\usepackage[T1]{fontenc}
\usepackage{mathtools}
\usepackage{color}
\usepackage{ulem}
\usepackage{tabu}
\usepackage{multirow}
\usepackage{rotating}

\usepackage[outline]{contour}
\contourlength{1.2pt}

\usepackage{tikz}
\usepackage{graphics}
\usepackage{xcolor}

\usepackage{pgfplots}
\usepackage{pgfplotstable}

\usepackage[at]{easylist}

\DeclareMathOperator{\sign}{sign}

\newcommand{\insertTitle}[6]{
\begin{titlepage}
	\begin{center}
    	\large
		Министерство науки и высшего образования Российской Федерации
		
		Новосибирский государственный технический университет
		%\vspace{0.25cm}
		\vfill
		{\textbf #1}
		
		Лабораторная работа №#2
		\vfill
	\end{center}
	
	\begin{tabular}{ m{7em}  m{7em} }
	Факультет: & ФПМИ \\ 
	Группа: & #3 \\  
	Студент: & #4 \\
	Вариант: & #5
	\end{tabular}
	\vfill

\begin{center}
Новосибирск

#6
\end{center}
\end{titlepage}
}

\usepackage[utf8]{inputenc}
\usepackage{listings}
\usepackage{color}
 
\definecolor{codegreen}{rgb}{0,0.6,0}
\definecolor{codegray}{rgb}{0.5,0.5,0.5}
\definecolor{codepurple}{rgb}{0.58,0,0.82}
\definecolor{backcolour}{rgb}{0.95,0.95,0.95}
 
\lstdefinestyle{mystyle}{
    backgroundcolor=\color{backcolour},   
    commentstyle=\color{codegreen},
    keywordstyle=\color{blue},
    numberstyle=\tiny\color{codegray},
    stringstyle=\color{codepurple},
    basicstyle=\ttfamily,
    breakatwhitespace=false,         
    breaklines=true,                 
    captionpos=b,                    
    keepspaces=true,                 
    numbers=left,                    
    numbersep=5pt,                  
    showspaces=false,                
    showstringspaces=false,
    showtabs=false,                  
    tabsize=4
}


\newcommand{\myCodeInput}[3]{
{\bf #2}
\lstinputlisting[language=#1]{#3}
}

%-------------------------------------------------------------------------------
%-------------------------------------------------------------------------------
%-------------------------------------------------------------------------------

\lstset{style=mystyle}

\begin{document}

\setlength{\abovedisplayskip}{0.1pt}
\setlength{\belowdisplayskip}{0.1pt}

%-------------------------------------------------------------------------------
\insertTitle{Уравнения математической физики}{4}{ПМ-63}{Кожекин М.В.}{4}{2019}


%-------------------------------------------------------------------------------
\section{Цель работы}
Изучить особенности реализации итерационных методов BCG, BCGSTAB, GMRES для СЛАУ с несимметричными разреженными матрицами. Исследовать влияние предобуславливания на сходимость этих методов.


%-------------------------------------------------------------------------------
\section{Задание}
1. Реализовать программу решения СЛАУ большой размерности в разреженном строчно-столбцовом формате в соответствии с заданием.

2. Протестировать разработанную программу на небольших матрицах.

3. Сравнить реализованный метод по вычисдительным затратам с методами используемыми в предыдущей лабораторной работе, на матрицах большой размерности, полученных в результате конечноэлементной аппроксимации в предыдущей работе.

{\bf Вариант 4}
Решить СЛАУ методом BCG без предобуславливания


%-------------------------------------------------------------------------------
\section{Анализ}
Классическая схема {\bf метода бисопряжённых градиентов} выглядит следующим образом:

Выбирается начальное приближение $\text{x}^0$ и полагается

\[ r^0 = b - Ax^0 \]
\[ p^0 = r^0 \]
\[ z^0 = r^0 \]
\[ s^0 = r^0 \]

Далее для k=1,2,... производятся следующие вычисления:

\[ \alpha_k = \frac{( p^{k-1}, r^{k-1} )} {( s^{k-1}, Az^{k-1 }) } \]

\[ x^k = x^{k-1} + \alpha_k z^{k-1} \]
\[ r^k = r^{k-1} - \alpha_k Az^{k-1} \]
\[ p^k = p^{k-1} - \alpha_k A^{T}s^{k-1} \]

\[ \beta_k = \frac{( p^{k}, r^{k} )} {( p^{k-1}, r^{k-1} )} \]

\[ z^k = x^{k-1} + \beta_k z^{k-1} \]
\[ s^k = p^{k-1} + \beta_k s^{k-1} \]

%-------------------------------------------------------------------------------
\section{Исследования}


%-------------------------------------------------------------------------------
\section{Выводы}

%-------------------------------------------------------------------------------
\section{Исходный код программы}
\myCodeInput{c++}{head.h}{../head.h}
\myCodeInput{c++}{solver.h}{../solver.h}
\myCodeInput{c++}{solver.cpp}{../solver.cpp}
\myCodeInput{c++}{main.cpp}{../main.cpp}

\end{document}

