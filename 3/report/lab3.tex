\documentclass[12pt, a4paper]{article}
\usepackage[russian]{babel}
\usepackage{fontspec}
\setsansfont{Calibri}
\setmonofont{Consolas}
\setmainfont[
    Ligatures=TeX,
    Extension=.otf,
    BoldFont=cmunbx,
    ItalicFont=cmunti,
    BoldItalicFont=cmunbi,
]{cmunrm}
\usepackage{polyglossia}
\setdefaultlanguage{russian}
\setotherlanguage{english}


\usepackage{geometry}
\usepackage{pgfplotstable}

\geometry{
margin=2cm
}

% Создаем команду, чтобы переносить текст на новую строку внутри таблицы
\newcommand{\tcell}[2][l]{\begin{tabular}[#1]{@{}c@{}}#2\end{tabular}}

\usepackage{indentfirst}

\usepackage{arydshln}
\usepackage[fleqn]{amsmath}
\usepackage{xfrac}
\usepackage{esint}
\usepackage{amssymb}
\usepackage{mathbbol}
\usepackage[T1]{fontenc}
\usepackage{mathtools}
\usepackage{color}
\usepackage{ulem}
\usepackage{tabu}
\usepackage{multirow}
\usepackage{rotating}
\usepackage{enumitem}

\usepackage[outline]{contour}
\contourlength{1.2pt}

\usepackage{tikz}
\usepackage{graphics}
\usepackage{xcolor}

\usepackage{pgfplots}
\usepackage{pgfplotstable}

\usepackage[at]{easylist}

\DeclareMathOperator{\sign}{sign}

\newcommand{\insertTitle}[6]{
\begin{titlepage}
	\begin{center}
    	\large
		Министерство науки и высшего образования Российской Федерации
		
		Новосибирский государственный технический университет
		%\vspace{0.25cm}
		\vfill
		{\textbf #1}
		
		Лабораторная работа №#2
		\vfill
	\end{center}
	
	\begin{tabular}{ m{7em}  m{7em} }
	Факультет: & ФПМИ \\ 
	Группа: & #3 \\  
	Студент: & #4 \\
	Вариант: & #5
	\end{tabular}
	\vfill

\begin{center}
Новосибирск

#6
\end{center}
\end{titlepage}
}




\usepackage{slashbox}
\newcommand{\inputTable}[1]{
\begin{center}
\noindent\footnotesize\pgfplotstabletypeset[
	columns={a,$1$,$x$,$x^2$,$x^3$,$x^4$,$x^5$,$sin(x)$,$e^x$},
	columns/a/.style={string type, column name={\backslashbox{$u_s(x, t)$}{$u_c(x, t)$}}},
	columns/$1$/.style={string type},
	columns/$x$/.style={string type},
	columns/$x^2$/.style={string type},
	columns/$x^3$/.style={string type},
	columns/$x^4$/.style={string type},
	columns/$x^5$/.style={string type},
	columns/$sin(x)$/.style={string type},
	columns/$e^x$/.style={string type, column name={$\sin u$}, column type/.add={}{|},},
	every head row/.style={before row=\hline,after row=\hline}, 
	every last row/.style={after row=\hline},
	column type/.add={|},
	col sep=tab,
]{#1}
\end{center}
}


\newcommand{\inputTableConvergence}[1]{
\noindent{\scriptsize\pgfplotstabletypeset[
	columns={i, nodes, iters, norm},
	columns/i/.style={string type},
	columns/nodes/.style={string type},
	columns/iters/.style={string type},
	columns/norm/.style={string type, column type/.add={}{|},},
	every head row/.style={before row=\hline,after row=\hline}, 
	every last row/.style={after row=\hline},
	column type/.add={|},
	col sep=tab,
]{#1}}
}

\newcommand{\inputTableConvergencecopy}[1]{
\[\begin{tabular}{ | с | с | с | m{11em}  m{8em} m{8em} }
	\hline
	\backslashbox{время}{пространство} & равномерное & не равномерное \\ \hline
	равномерное & \inputTableConvergence{#1.11.txt} & \inputTableConvergence{#1.01.txt} \\ \hline
	не равномерное & \inputTableConvergence{#1.10.txt} & \inputTableConvergence{#1.00.txt} \\
	\hline
\end{tabular}\]
}




\usepackage[utf8]{inputenc}
\usepackage{listings}
\usepackage{color}
 
\definecolor{codegreen}{rgb}{0,0.6,0}
\definecolor{codegray}{rgb}{0.5,0.5,0.5}
\definecolor{codepurple}{rgb}{0.58,0,0.82}
\definecolor{backcolour}{rgb}{0.85,0.85,0.85}
 
\lstdefinestyle{mystyle}{
	backgroundcolor=\color{backcolour},   
	commentstyle=\color{codegreen},
	keywordstyle=\color{blue},
	basicstyle=\fontsize{10}{12}\selectfont\ttfamily,
   	numberstyle=\tiny\color{codegray},
	stringstyle=\color{codepurple},
    	breakatwhitespace=false,         
	breaklines=true,                 
	captionpos=b,                    
	keepspaces=true,                 
	numbers=left,                    
	numbersep=5pt,                  
	showspaces=false,                
	showstringspaces=false,
	showtabs=false,                  
	tabsize=4
}
\lstset{ style=mystyle}


\newcommand{\myCodeInput}[3]{
{\bf #2}
\lstinputlisting[language=#1]{#3}
}



%-------------------------------------------------------------------------------
%-------------------------------------------------------------------------------
%-------------------------------------------------------------------------------

\begin{document}

\setlength{\abovedisplayskip}{1pt}
\setlength{\belowdisplayskip}{1pt}

%-------------------------------------------------------------------------------
\insertTitle{Уравнения математической физики}{3}{ПМ-63}{Кожекин М.В.}{1}{2019}


%-------------------------------------------------------------------------------
\section{Цель работы}
Разработать программу решения гармонической задачи методом конечных элементов. Сравнить прямой и итерационные методы решения получаемой в результате конечноэлементной аппроксимации СЛАУ.


%-------------------------------------------------------------------------------
\section{Задание}

1.	Выполнить конечноэлементную аппроксимацию исходного уравнения в соответствии с заданием. Получить формулы для вычисления компонент матрицы ${\bf A}$ и вектора правой части ${\bf b}$. 

2.	Реализовать программу решения гармонической задачи с учетом следующих требований:
\begin{itemize}[noitemsep]
\item язык программирования С++ или Фортран;
\item предусмотреть возможность задания неравномерной сетки по пространству, разрывность параметров уравнения по подобластям, учет краевых условий;
\item матрицу хранить в разреженном строчно-столбцовом формате с возможностью перегенерации ее в профильный формат; 
\item реализовать (или воспользоваться реализованными в курсе «Численные методы») методы решения СЛАУ: итерационный – локально-оптимальную схему или метод сопряженных градиентов для несимметричных матриц с предобусловливанием и прямой – LU-разложение или его модификации [2, с. 871; 3].
\end{itemize}

3.	Протестировать разработанную программу на полиномах первой степени.

4.	Исследовать реализованные методы для сеток с небольшим количеством узлов 500 – 1000 и большим количеством узлов – примерно 20 000 – 50 000 для различных значений параметров $ 10^{-4} \leq \omega \leq 10^{9} $, $ 10^{2} \leq \lambda \leq 8\cdot10^{5} $, $ 0 \leq \sigma \leq 10^{8} $, $ 8.81 \cdot 10^{-12} \leq \chi \leq 10^{-10} $. Для всех решенных задач сравнить вычислительные затраты, требуемые для решения СЛАУ итерационным и прямым методом.



{\bf Вариант 1:}
Решить одномерную гармоническую задачу в декартовых координатах, базисные функции - линейные.


%-------------------------------------------------------------------------------
\section{Анализ}

Рассмотрим задачу для уравнения:

\begin{aligned}
\chi \frac{d^{2}u}{dt^{2}} + \sigma \frac{du}{dt} - div(\lambda grad(u)) = f \\[5pt]
\end{aligned}

Решение данного уравния u и его правая часть f представимы в виде:

\begin{aligned}
u(x,y,t) = u^s sin \omega t + u^c cos \omega t \\[5pt]
f(x,y,t) = f^s sin \omega t + f^c cos \omega t \\[5pt]
\end{aligned}

Значит исходное уравнение можно привести к системе уравнений

\begin{aligned}
- div(\lambda grad(u^s)) - \omega \sigma u^c  - \omega^2 \chi u^s = f \\[5pt]
- div(\lambda grad(u^c)) - \omega \sigma u^s  - \omega^2 \chi u^c = f \\[5pt]
\end{aligned}


\begin{aligned}
p_{ij}(q_s) &= \int_{\Omega}{\Bigl( \lambda grad\psi_i grad\psi_j - \omega^2 \chi 
\psi_i \psi_j \Bigr) d\Omega} \\[6pt]
c_{ij}(q_s) &= \omega \int_{\Omega}{\sigma \psi_i \psi_j d\Omega}  \\[6pt] 
\end{aligned}

Матрица конечноэлементной СЛАУ будет иметь следующую структуру:

\begin{aligned}
\begin{pmatrix}
	p_{00} & -c_{00} & p_{01} & -c_{01} \\
	c_{00} & p_{00} & c_{01} & p_{01} \\
	p_{10} & -c_{10} & p_{11} & -c_{11} \\
	c_{10} & p_{10} & c_{11} & p_{11} \\
 \end{pmatrix} \\[6pt]
\end{aligned}
%-------------------------------------------------------------------------------
%-------------------------------------------------------------------------------
%-------------------------------------------------------------------------------
\par\noindent\rule{\textwidth}{0.4pt}

Выведем формулы для локальных матриц массы, жёсткости и вектора правой части.

\begin{aligned}
\frac{du}{dx} &= \frac{d\sum_{k = 0}^{1}{q_k \psi_k}}{dx} = q_0 \frac{d \psi_0}{dx} + q_1 \frac{d \psi_1}{dx} = -\frac{1}{h}q_0 + \frac{1}{h}q_1 = \frac{q_1 - q_0}{h} \\[6pt]
G_{i,j}&=\int_{\Omega}{\lambda(\frac{du}{dx}) grad\psi_i grad\psi_j d\Omega}  \\[6pt]
G_{0,0} &= \sum_{k=0}^{1} \int_{\Omega}{\lambda(\frac{q_1 - q_0}{h}) \psi_k grad\psi_0 grad\psi_0 d\Omega} = \\[6pt]
&= \frac{1}{h}\sum_{k=0}^{1} \int_{\Omega}{\lambda(\frac{q_1 - q_0}{h}) \psi_k d\Omega} = \\[6pt]
&= \frac{\lambda_0(\frac{q_1-q_0}{h}) + \lambda_1(\frac{q_1-q_0}{h})}{2h} =  G_{1,1} \\[6pt]
G_{0,1} &= \sum_{k=0}^{1} \int_{\Omega}{\lambda(\frac{q_1 - q_0}{h}) \psi_k grad\psi_0 grad\psi_1 d\Omega} = \\[6pt]
&= -\frac{1}{h}\sum_{k=0}^{1} \int_{\Omega}{\lambda(\frac{q_1 - q_0}{h}) \psi_k  d\Omega} = \\[6pt]
&= -\frac{\lambda_0(\frac{q_1-q_0}{h}) + \lambda_1(\frac{q_1-q_0}{h})}{2h} = G_{1,0}  \\[6pt]
G &= \frac{\lambda_0(\frac{q_1-q_0}{h}) + \lambda_1(\frac{q_1-q_0}{h})}{2h}\begin{pmatrix} 1 & -1 \\ -1 & 1 \end{pmatrix} \\[6pt]
\end{aligned}




%-------------------------------------------------------------------------------
%-------------------------------------------------------------------------------
%-------------------------------------------------------------------------------
\par\noindent\rule{\textwidth}{0.4pt}

\begin{aligned}
M_{i,j} &= \frac{\sigma}{\Delta t_s}\int_{\Omega}{\psi_i\psi_jd\Omega} \\[6pt]
M_{0,0} &= \frac{\sigma}{\Delta t_s}\int_{\Omega}{\psi_0\psi_0d\Omega} = \frac{\sigma h}{\Delta t_s} \int_0^1{\xi^2 d\xi } = \frac{\sigma h}{\Delta t_s} \frac{\xi^3}{3}\Bigr|_{0}^{1} = \frac{\sigma h}{3 \Delta t_s} = M_{1,1} \\[6pt]
M_{0,1} &= \frac{\sigma}{\Delta t_s}\int_{\Omega}{\psi_0\psi_1d\Omega} = \frac{\sigma h}{\Delta t_s} \int_0^1{\xi(1-\xi) d\xi } = \frac{\sigma h}{\Delta t_s} \bigl( \frac{\xi^2}{2} - \frac{\xi^3}{3} \bigr) \Bigr|_{0}^{1} = \frac{\sigma h}{6 \Delta t_s} = M_{1,0} \\[6pt]
M &= \frac{\sigma h}{6 \Delta t_s}\begin{pmatrix} 2 & 1 \\ 1 & 2 \end{pmatrix} \\[6pt]
\end{aligned}





%-------------------------------------------------------------------------------
%-------------------------------------------------------------------------------
%-------------------------------------------------------------------------------
\par\noindent\rule{\textwidth}{0.4pt}
\begin{aligned}
b_i &= \int_{\Omega}{f_s\psi_i d\Omega}    +    \frac{1}{\Delta t_s}\int_{\Omega}{\sigma u_{q-1}^h \psi_id\Omega} \left| u_{q-1}{h} = \sum_{k=0}^1{q_{k,s-1}\psi_k} \right| \\[6pt]
b_0 &= \sum_{k=0}^{1} \int_{\Omega} {f_k \psi_k \psi_0 d\Omega}    +    \frac{\sigma}{\Delta t_s}\sum_{k=0}^{1} \int_{\Omega}{q_{k,q-1} \psi_k \psi_0 d\Omega} \\[6pt]
&= \Bigl[ f_0 \int_{\Omega}{\psi_0 \psi_0 d\Omega} + f_1 \int_{\Omega}{\psi_1 \psi_0 d\Omega} \Bigr]    +    \frac{\sigma}{\Delta t_s}\Bigl[ q_{0,s-1}\int_{\Omega}{\psi_0 \psi_0 d\Omega} + q_{1,s-1}\int_{\Omega}{\psi_1 \psi_0 d\Omega} \Bigr] \\[6pt]
&= h \Bigl[ f_0 \int_{0}^{1}{\xi^2 d\xi} + f_1 \int_{0}^{1}{(1-\xi) \xi d\xi} \Bigr]    +    \frac{\sigma}{\Delta t_s}\Bigl[ q_{0,s-1}\int_{0}^{1}{\xi^2 d\xi} + q_{1,s-1}\int_{0}^{1}{(1-\xi) 
\xi d\xi} \Bigr] \\[6pt]
&= h \Bigl[ f_0 \frac{\xi^3}{3} \Bigr|_{0}^{1} + f_1 \bigl( \frac{\xi^2}{2} - \frac{\xi^3}{3} \bigr) \Bigr|_{0}^{1} \Bigr]    +    \frac{\sigma}{\Delta t_s} \Bigl[ q_{0,s-1}\frac{\xi^3}{3} \Bigr|_{0}^{1} + q_{1,s-1}\bigl( \frac{\xi^2}{2} - \frac{\xi^3}{3} \bigr) \Bigr|_{0}^{1} \Bigr] \\[6pt]
&= h \Bigl[ f_0\frac{1}{3} + f_1\frac{1}{6} \Bigr] + \frac{\sigma}{\Delta t_s} \Bigl[ \frac{1}{3}q_{0,s-1}    +    \frac{1}{6}q_{1,s-1} \Bigr] \\[6pt]
&= \frac{h}{6} \Bigl[ 2f_0 + f_1 \Bigr] + \frac{\sigma}{6 \Delta t_s} \Bigl[ 2q_{0,s-1} + q_{1,s-1} \Bigr] \\[6pt]
b_1 &= \sum_{k=0}^{1} \int_{\Omega} {f_k \psi_k \psi_1 d\Omega}    +    \frac{\sigma}{\Delta t_s}\sum_{k=0}^{1} \int_{\Omega}{q_{k,q-1} \psi_0 \psi_1 d\Omega} = \\[6pt]
&= \Bigl[ f_0 \int_{\Omega}{\psi_0 \psi_1 d\Omega} + f_1 \int_{\Omega}{\psi_1 \psi_1 d\Omega} \Bigr]    +    \frac{\sigma}{\Delta t_s}\Bigl[ q_{0,s-1}\int_{\Omega}{\psi_0 \psi_1 d\Omega} + q_{1,s-1}\int_{\Omega}{\psi_1 \psi_1 d\Omega} \Bigr] = \\[6pt]
&= h \Bigl[ f_0 \int_{0}^{1}{\xi (1-\xi) d\xi} + f_1 \int_{0}^{1}{(1-\xi)^2 d\xi} \Bigr]    +    \frac{\sigma}{\Delta t_s}\Bigl[ q_{0,s-1}\int_{0}^{1}{\xi (1-\xi) d\xi} + q_{1,s-1}\int_{0}^{1}{(1-\xi)^2 d\xi} \Bigr] = \\[6pt]
&=h \Bigl[ f_0 \bigl( \frac{\xi^2}{2} - \frac{\xi^3}{3} \bigr) \Bigr|_{0}^{1} + f_1 (1 - \xi)^3 \Bigr|_{0}^{1} \Bigr]    +    \frac{\sigma}{\Delta t_s} \Bigl[ q_{0,s-1}\bigl( \frac{\xi^2}{2} - \frac{\xi^3}{3} \bigr) \Bigr|_{0}^{1} + q_{1,s-1}(1 - \xi)^3 \Bigr|_{0}^{1} \Bigr] =  \\[6pt]
&= \frac{h}{6}\Bigl[f_0\frac{1}{6}+f_1\frac{1}{3}\Bigr]    +    \frac{\sigma}{\Delta t_s} \Bigl[ \frac{1}{6}q_{0,s-1} + \frac{1}{3}q_{1,s-1} \Bigr] \\[6pt]
&=\frac{h}{6}\Bigl[f_0+2f_1\Bigr]    +    \frac{\sigma}{6 \Delta t_s} \Bigl[ q_{0,s-1} + 2 q_{1,s-1} \Bigr] \\[6pt]
b &= \frac{hx}{6}\begin{pmatrix} 2f_0+f_1 \\ f_0+2f_1\end{pmatrix} + \frac{\sigma}{6 \Delta t_s} \begin{pmatrix} 2q_{0,s-1} + q_{1,s-1} \\  q_{0,s-1} + 2 q_{1,s-1} \end{pmatrix} \\[6pt]
\end{aligned}
\par\noindent\rule{\textwidth}{0.4pt}

В итоге:

\begin{aligned}
G &= \frac{\lambda_0(\frac{q_1-q_0}{h}) + \lambda_1(\frac{q_1-q_0}{h})}{2h}\begin{pmatrix} 1 & -1 \\ -1 & 1 \end{pmatrix} \\[6pt]
M &= \frac{\sigma h}{6 \Delta t_s}\begin{pmatrix} 2 & 1 \\ 1 & 2 \end{pmatrix} \\[6pt]
b &= \frac{hx}{6}\begin{pmatrix} 2f_0+f_1 \\ f_0+2f_1\end{pmatrix} + \frac{\sigma}{6 \Delta t_s} \begin{pmatrix} 2q_{0,s-1} + q_{1,s-1} \\  q_{0,s-1} + 2 q_{1,s-1} \end{pmatrix}
\end{aligned} \\[6pt]
\par\noindent\rule{\textwidth}{0.4pt}




\section{Точность для разных функций u и $\lambda$}
%-------------------------------------------------------------------------------

В ходе следующего исследования использовались следующие параметры:

$\varepsilon = 1e-7$

$\sigma = 1$

$maxiter = 1000$

$\text{Область пространства } \Omega = [0, 1]$

$\text{Время задано на отрезке [0, 1]}$

$\text{Первоначальное число узлов 11, а конечных элементов 10}$

$\text{Для неравномерных сеток по времени и пространству коэффициент k=1.1}$

\inputTable{table.txt}
%-------------------------------------------------------------------------------
\subsection{Вывод}

Как видно из таблицы метод начинает сходиться хуже при повышении степени полинома. Если же функция $\lambda$ будет зависеть не от $\frac{du}{dx}$, а просто от u, то сходимость будет куда выше . Если функция $\lambda$ гармоническая (в нашем случае sin(u)), то метод работает хуже, хотя вообще он не должен сходиться. 

Также стоит отметить, что и скорость программы в варианте 7 заметно ниже варианта 5. Рещение сходится медленно.



%-------------------------------------------------------------------------------
\section{Точность решения при дроблении сетки}

В ходе следующего исследования использовались следующие параметры:

$\varepsilon = 1e-22$

$\sigma = 1$

$maxiter = 1000$, т.к. повышение этого числа не приводит к должному результату, а лишь занимает процессорное время

$\text{Область пространства } \Omega = [0, 1]$

$\text{Время задано на отрезке [0, 1]}$

$\text{Первоначальное число узлов 11, а конечных элементов 10}$

$\text{Для неравномерных сеток по времени и пространству коэффициент k=1.1}$


\[ u_s = x, u_c = -2 \cdot x \]
\inputTableConvergencecopy{file_u0}
\[ u_s = x, u_c = -2 \cdot x \]
\inputTableConvergencecopy{file_u1}
\[ u_s = x^2, u_c = -2 \cdot x^2 \]
\inputTableConvergencecopy{file_u2}
\[ u_s = x^3, u_c = -2 \cdot x^3 \]
\inputTableConvergencecopy{file_u3}
\[ u_s = x^4, u_c = -2 \cdot x^4 \]
\inputTableConvergencecopy{file_u4}
\[ u_s = x^5, u_c = -2 \cdot x^5 \]
\inputTableConvergencecopy{file_u5}
\[ u_s = sin(x), u_c = -2 \cdot sin(x) \]
\inputTableConvergencecopy{file_u6}
\[ u_s = e^x, u_c = -2 \cdot e^x \]
\inputTableConvergencecopy{file_u7}

%-------------------------------------------------------------------------------
\subsection{Вывод}

Т.к. {\bf порядок сходимости} - это степень того, насколько сильно увеличивается точность при дроблении сетки. Он определяется из степени x.

Исходя из исследований можно заметить, что порядок сходимости $\frac{1}{3}$



%-------------------------------------------------------------------------------
\section{Исходный код программы}
%\myCodeInput{c++}{head.h}{../head.h}
%
%\myCodeInput{c++}{grid.h}{../grid.h}
%\myCodeInput{c++}{grid.cpp}{../grid.cpp}
%
%\myCodeInput{c++}{fem.h}{../fem.h}
%\myCodeInput{c++}{fem.cpp}{../fem.cpp}
%
%\myCodeInput{c++}{solver.h}{../solver.h}
%\myCodeInput{c++}{solver.cpp}{../solver.cpp}
%
%\myCodeInput{c++}{main.cpp}{../main.cpp}

\end{document}
